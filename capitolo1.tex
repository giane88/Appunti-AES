\label{capitolo1}
\section{Introduzione}I sistemi che generano, distribuiscono e consumano esergia sono diventati sempre pi� complessi ed articolati a causa di diversi tipi di sorgente dai quali l'energia viene ricavata (rinnovabili e non), diversi metodi di generazione, un mercato complesso ed in rapida evoluzione.\\
La progettazione di sistemi energetici richiede perci� sempre pi� la figura di un ingegnere automatico che abbia una visione del sistema a pi� livelli, permetta una sempre pi� alta integrazione e sia in grado di soddisfare le nuove esigenze.\\
Scopo del corso � quello di fornire una visione a livello di sistema dei problemi di controllo analizzando le soluzione adottate e gli sviluppi futuri senza entrare troppo nel dettaglio dei generatori e utilizzatori.
\subsection{L'automazione nei sistemi energetici}
I principi di automazione nei sistemi energetici li troviamo praticamente ovunque, nei refrigeratori, nelle caldaie, nelle lavatrici; pi� in generale nei generatori e nei consumatori.
A qualsiasi livello di progetto grandi componenti, macchinari industriali, piani energetici.\\
Praticamente l'automazione nei sistemi energetici � ovunque si abbia una trasformazione di energia, dove la parola "trasformazione" assume un significato diverso a seconda del contesto nel quale lavoriamo, ad esempio parliamo di "generazione" nel caso produzione di energia, "consumo" nel caso di utilizzo di energia e di "trasporto".\\
Ogni oggetto (o aggregazione di oggetti) pu� essere controllato per raggiungere degli obiettivi locali, ma bisogna sempre tener presente che ogni azione su un oggetto influenza l'intero sistema.\\
Nonostante tutto non si pu� definire una gerarchia tra i sistemi in quanto questa caratteristica richiederebbe un controllore enorme per eseguire il lavoro.\\
Possiamo comunque strutturare il sistema capendo quali sono i problemi rilevanti e se e come possono essere controllati ragionevolmente.
Un problema pu� essere dominato se la sua estensione � finita e pu� essere descritta da condizioni di confine specifiche ed � possibile trovare per esso almeno una soluzione di complessit� accettabile.\\
Per descrivere e capire il problema abbiomo bisogno di un approccio sistematico nel quale i componenti sono descritti a diversi livelli di dettaglio, preservano le loro interfacce con gli altri componenti e sono il pi� possibile indipendenti da come essi sono connessi con il resto del sistema.
\subsection{Definizioni di base}
Di seguito alcune definizione di base prima di cominciare.
\begin{description}
\item[Primary-Secondary energy(PE-SE)]:
\begin{itemize}
\item Primary Energy: � quell'energia che pu� essere ricavata direttamente in natura(gas, vento, sole);
\item Secondary Energy: � quell'energia ricavata trasformando l'energia primaria in una forma pi� facile da utilizzare, immagazzinare e trasportare.
\end{itemize}
\item[Renewable-Non Renewable Energy Sources (RES-NRES)]:
\begin{itemize}
\item Energia rinnovabile � ottenuta da una fonte di energia praticamente inesauribile.
\item Energia non rinnovabile richiede un consumo di "carburante" e il rilascio di inquinanti (es. petrolio, gas).
\end{itemize}
\item[Energy intensity]:Quantit� di energia per unit�; ad esempio per unit� prodotta o per servire una determinata richiesta.
\item[Energy conservation]: Riduzione in termini assoluti del consumo di energia.
\item[Energy efficenty]: Riduzione di EI preservando la quantit� di risultato.
\end{description}

\subsection{Alcuni problemi di controllo}
\begin{description}
\item[Controllo di generatori (PE$\rightarrow$ SE)]: minimizzare il consumo di carburante nel caso di NRES o massimizzare rendimento (RES).
\item[Controllo consumatori (SE$\rightarrow$fianl user)]:mantenere le condizioni ottimali (temperatura della stanza)
\end{description}
